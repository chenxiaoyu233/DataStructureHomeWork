\documentclass{article}
\usepackage{amsmath}
\usepackage{ctex}
\usepackage{enumerate}

\usepackage{algorithm}
\usepackage{algorithmicx}
\usepackage{algpseudocode}

\title{数据结构第二次作业}
\author{陈小羽\ 2016060106020}
\date{}

\begin{document}
\maketitle
\tableofcontents
\section{题目要求}
	\begin{itemize}
		\item 从文件中读取迷宫数据,寻找并打印路径通路,储存迷宫数据到文件。
		\item 寻找多入口多出口地图的所有通路。
		\item 找到入口到出口最短的一条通路。
		\item 自动生成迷宫地图。
	\end{itemize}
\section{寻找路径}
	\subsection{算法选择}
		\paragraph{}
			按照题目的要求,我们需要找到\textbf{起点集合}到\textbf{终点集合}
			之间的一条最短的通路。因为迷宫中相邻的两个格子之间的距离始终为1,
			所以我们可以简单的使用\textbf{宽度优先搜索(BFS)}来完成任务。
		\subsubsection{算法描述}
			\paragraph{}
				BFS算法有些类似现实中的声纳探测器。都从一点开始,向外辐射,寻找
				目标。不同的是,声纳探测器需要接受返回的声波才能确定是否发现
				目标,而BFS算法找到目标之后可以直接报告。
			\paragraph{维护波}
				为了实现BFS算法,我们需要模拟现实中的波。因为我们没有必要走回头路,
				所以我们只需要维护整个波的\textbf{波前}, 也就是走在最前面的\textbf{波面}。
			\paragraph{队列}
				因为我们知道,队列具有\textbf{先进先出}的优良性质。所以我们可以使用队列
				来维护\textbf{波前}。具体的操作如下:
				\begin{enumerate}[1]
					\item 将所有的起点入队。
					\item 将对首的位置出队。然后访问其周围的位置。
					\item 如果波访问到了一个位置, 且这个位置之前没有访问过,我们就将这个位置入队。
				\end{enumerate}
				通过这种方式,我们就用程序模拟了现实中的波。其中,队列中的元素就是波的波前。
				我们可以发现,队列中的任意两个元素,到起点的距离之差不会超过1。
			\paragraph{寻路算法}
				我们将所有的起点入队,然后用BFS算法模拟若干个波前。这样,我们遇到的第一个终点,
				就一定是所有的起点到终点中最近的那一个。这样,我们就完成了题目的要求。
				我们在\textbf{算法1}中给出了伪代码。
				\begin{algorithm}[!h]
					\caption{BFS寻路算法}
					\begin{algorithmic}[1]
						\Function {BFS} {起点集合$S$, 终点集合$T$, 地图$Map$}
							\For {地图中的每一个可能的位置$w_i$}
								\Comment {初始化}
								\State $Distance[w_i] \gets \infty$
							\EndFor
							\State $Queue = \emptyset$
							\Comment {初始化}
							\For {每一个在起点集合$S$中的元素$s_i$}
								\Comment {起点入队}
								\State {$Queue.tail \gets s_i$}
								\State {$Distance[s_i] \gets 0$}
							\EndFor
							\While{$Queue$非空}
								\State {$Front \gets Queue.pop$}
								\Comment {对首元素出队}
								\If {$Front \in T$} 
									\Comment {找到答案,直接返回}
									\State{\Return {$Distance[Front]$}}
								\EndIf
								\For {$Front$ 的每个邻近位置 $p$}
									\If {$p$没有障碍物且之前没有访问过}
										\Comment{扩展波前}
										\State {$Queue.tail \gets p$}
										\State {$Distance[p] \gets Distance[Front] + 1$}
									\EndIf
								\EndFor
							\EndWhile
							\State{\Return $\infty$}
							\Comment{没有找到答案}
						\EndFunction
					\end{algorithmic}
				\end{algorithm}
		\subsubsection{算法分析}
			\paragraph{}
				因为每一个点最多访问一遍,所以算法的时间复杂度是$T \in \mathcal{O}(nm)$的,其中$n, m$分别表示
				迷宫地图的长和宽。空间复杂度和时间复杂度一致。
\section{生成地图}
	\subsection{算法选择}
		\paragraph{}
			按照ppt中的提示,我们可以将迷宫中的所有可以到达的点做成一棵连通的树。
			实现这个操作的一个很棒的算法就是\textbf{克鲁斯卡尔算法}
		\subsubsection{算法描述}
			\paragraph{}
				主要的操作很简单,一开始,所有可以到达的点都是不连通的。然后我们考虑不断的
				打通一些墙,使得最终,所有的这些点都是连通的,且形成一棵树。为此,我们需要
				维护哪些点已经连通了,因为在这些点之间继续加边的话,会形成环。
			\paragraph{维护连通集合}
				我们使用一种被称为\textbf{并查集}的数据结构来维护连通集合。并查集通过树的父亲表示法,
				表示了一个森林。其中,每个树的根节点的父亲就是这个点自己。这样,如果两个点
				在这个森林的同一棵树中,我们就说,这两个点在同一个集合中。我们可以快速的判断每个点
				所在树的根节点,并通过这个来判断两个点是否在同一棵树中。我们在\textbf{算法2}中给出了
				这种算法的伪代码。
				\begin{algorithm}
					\caption{并查集}
					\begin{algorithmic}[1]
						\Function {Initial} {}
							\Comment {初始化}
							\For {每个点 $p_i$}
								\State {$father[p_i] = p_i$}
							\EndFor
						\EndFunction
						\Function {Find} {点$p$}
							\Comment {找到点$p$所在树的根节点}
							\If {$father[p] = p$}
								\State{ \Return {$p$}}
							\Else 
								\State{ \Return {$father[p] \gets $ FIND($father[p]$)}}
								\Comment {路径压缩}
							\EndIf
						\EndFunction
						\Function {union} {点$p_x$,$p_y$}
							\Comment {将$p_x$和$p_y$加入同一个集合}
							\State {$f_x \gets $ FIND($p_x$)}
							\State {$f_y \gets $ FIND($p_y$)}
							\If {$f_x \not= f_y$}
								\Comment {没有这个if有可能成环,产生死循环}
								\State {$father[f_x] = f_y$}
							\EndIf
						\EndFunction
					\end{algorithmic}
				\end{algorithm}
		\subsubsection{算法分析}
			\paragraph{}
				执行算法时,我们枚举了每条边,且并查集每次操作的复杂度可以看作常数,所以,
				总的复杂度和边的数目同阶,故时间和空间复杂度都是$\mathcal{O}(nm)$的,其中,
				$m,n$分别表示地图的长和宽。
\section{结论}
	\subsection{结果}
		\paragraph{} 
			程序运行正常。
	\subsection{一些值得改进的地方}
		\paragraph{$\cdots$}
\end{document}
